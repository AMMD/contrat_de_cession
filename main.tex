% Contrat de session
%
% Auteur  : Gregory DAVID
%%
\documentclass[a4paper,10pt]{article}

% Paquets
\usepackage[francais]{babel}
\usepackage{ucs}
\usepackage[utf8]{inputenc}
\usepackage[T1]{fontenc}
\usepackage[margin={1cm,1.5cm},includehead,includefoot]{geometry}
\usepackage{graphicx}
\usepackage{fancyhdr}
\usepackage{lastpage}
\usepackage{eurosym}
\usepackage{ifthen}
\usepackage[french]{varioref}

\newcommand{\VILLEAMMD}{CONNERR\'E}
\newcommand{\ADRAMMD}{Mairie,\\ Rue de l'Abreuvoir}
\newcommand{\CPAMMD}{72160}
\newcommand{\TELAMMD}{(+33)(0) 2 43 82 33 49}
\newcommand{\REPRESENTANTAMMD}{Dimitri EVEN}
\newcommand{\QUALITEREPRESENTANT}{président}
\newcommand{\SIRETAMMD}{SIRET : 50334086100023}
\newcommand{\HorF}{}
\newcommand{\AMMDlong}{Amicale du Mekanik Metal Disco}

\newcommand{\NUMBERCONTRAT}{20140824-BF}
\newcommand{\BANDNAME}{Barons Freaks}
\newcommand{\RAISONSOCIALE}{Le joli Collectif / Théâtre de Poche - Hédé}
\newcommand{\ADRESSE}{10, place de la Mairie\newline 35360 HÉDÉ}
\newcommand{\TELEPHONE}{09 81 83 97 16}
\newcommand{\SIRET}{SIRET : 45405131900035\newline NAF : 9002Z}
\newcommand{\ENTREPRENEUR}{Licences : 1-1054665 / 2-1054666 / 3-1054667}
\newcommand{\CAT}{}
\newcommand{\REPRESENTANT}{Christelle PICAULT}
\newcommand{\QUALITE}{Présidente}
\newcommand{\PAYS}{France}
\newcommand{\NOMSPECTACLE}{Parade des Barons Freaks}
\newcommand{\DUREESPECTACLE}{60'} % CHANGE-ME
\newcommand{\NBBANDPERS}{4} % CHANGE-ME
\newcommand{\LICENCETYPE}{Licence Art Libre 1.3 (LAL 1.3)}
\newcommand{\INPUTLICENCE}{\emph{Licence Art Libre 1.3 (LAL 1.3)}

\subsection{Pr\'eambule}

Avec la Licence Art Libre, l'autorisation est donn\'ee de copier, de diffuser et de transformer librement les \\oe  uvres dans le respect des droits de l'auteur.

Loin d'ignorer ces droits, la Licence Art Libre les reconna\^it et les prot\`ege. Elle en reformule l'exercice en permettant \`a tout un chacun de faire un usage cr\'eatif des productions de l'esprit quels que soient leur genre et leur forme d'expression.

Si, en r\`egle g\'en\'erale, l'application du droit d'auteur conduit \`a restreindre l'acc\`es aux \oe uvres de l'esprit, la Licence Art Libre, au contraire, le favorise. L'intention est d'autoriser l'utilisation des ressources d'une \oe uvre ; cr\'eer de nouvelles conditions de cr\'eation pour amplifier les possibilit\'es de cr\'eation. La Licence Art Libre permet d'avoir jouissance des \oe uvres tout en reconnaissant les droits et les responsabilit\'es de chacun.

Avec le d\'eveloppement du num\'erique, l'invention d'internet et des logiciels libres, les modalit\'es de cr\'eation ont \'evolu\'e : les productions de l'esprit s'offrent naturellement \`a la circulation, \`a l'\'echange et aux transformations. Elles se pr\^etent favorablement \`a la r\'ealisation d'\oe uvres communes que chacun peut augmenter pour l'avantage de tous.

C'est la raison essentielle de la Licence Art Libre : promouvoir et prot\'eger ces productions de l'esprit selon les principes du copyleft : libert\'e d'usage, de copie, de diffusion, de transformation et interdiction d'appropriation exclusive.

\subsection{D\'efinitions}

Nous d\'esignons par \og~\oe uvre~\fg, autant l'\oe uvre initiale, les \oe uvres cons\'equentes, que l'\oe uvre commune telles que d\'efinies ci-apr\`es :

\begin{description}
    \item[L'\oe uvre commune :]
Il s'agit d'une \oe uvre qui comprend l'\oe uvre initiale ainsi que toutes les contributions post\'erieures (les originaux cons\'equents et les copies). Elle est cr\'e\'ee \`a l'initiative de l'auteur initial qui par cette licence d\'efinit les conditions selon lesquelles les contributions sont faites.

    \item[L'\oe uvre initiale :]
C'est-\`a-dire l'\oe uvre cr\'e\'ee par l'initiateur de l'\oe uvre commune dont les copies vont \^etre modifi\'ees par qui le souhaite.

    \item[Les \oe uvres cons\'equentes :]
C'est-\`a-dire les contributions des auteurs qui participent \`a la formation de l'\oe uvre commune en faisant usage des droits de reproduction, de diffusion et de modification que leur conf\`ere la licence.

    \item[Originaux (sources ou ressources de l'\oe uvre) :]
Chaque exemplaire dat\'e de l'\oe uvre initiale ou cons\'equente que leurs auteurs pr\'esentent comme r\'ef\'erence pour toutes actualisations, interpr\'etations, copies ou reproductions ult\'erieures.

    \item[Copie :]
Toute reproduction d'un original au sens de cette licence.

\end{description}

\subsection{OBJET}
Cette licence a pour objet de d\'efinir les conditions selon lesquelles vous pouvez jouir librement de l'\oe uvre.

\subsection{L'\'ETENDUE DE LA JOUISSANCE}
Cette \oe uvre est soumise au droit d'auteur, et l'auteur par cette licence vous indique quelles sont vos libert\'es pour la copier, la diffuser et la modifier.

\subsubsection{LA LIBERT\'E DE COPIER (OU DE REPRODUCTION)}
Vous avez la libert\'e de copier cette \oe uvre pour vous, vos amis ou toute autre personne, quelle que soit la technique employ\'ee.

\subsubsection{LA LIBERT\'E DE DIFFUSER (INTERPR\'ETER, REPR\'ESENTER, DISTRIBUER)}
Vous pouvez diffuser librement les copies de ces \oe uvres, modifi\'ees ou non, quel que soit le support, quel que soit le lieu, \`a titre on\'ereux ou gratuit, si vous respectez toutes les conditions suivantes :
\begin{itemize}
\item joindre aux copies cette licence \`a l'identique ou indiquer pr\'ecis\'ement o\`u se trouve la licence ;
\item indiquer au destinataire le nom de chaque auteur des originaux, y compris le v\^otre si vous avez modifi\'e l'\oe uvre ;
\item indiquer au destinataire o\`u il pourrait avoir acc\`es aux originaux (initiaux et/ou cons\'equents).
\end{itemize}

Les auteurs des originaux pourront, s'ils le souhaitent, vous autoriser \`a diffuser l'original dans les m\^emes conditions que les copies.

\subsubsection{LA LIBERT\'E DE MODIFIER}
Vous avez la libert\'e de modifier les copies des originaux (initiaux et cons\'equents) dans le respect des conditions suivantes :
\begin{itemize}
\item celles pr\'evues \`a l'article 2.2 en cas de diffusion de la copie modifi\'ee ;
\item indiquer qu'il s'agit d'une \oe uvre modifi\'ee et, si possible, la nature de la modification ;
\item diffuser cette \oe uvre cons\'equente avec la m\^eme licence ou avec toute licence compatible ;
Les auteurs des originaux pourront, s'ils le souhaitent, vous autoriser \`a modifier l'original dans les m\^emes conditions que les copies.
\end{itemize}

\subsection{DROITS CONNEXES}
Les actes donnant lieu \`a des droits d'auteur ou des droits voisins ne doivent pas constituer un obstacle aux libert\'es conf\'er\'ees par cette licence.
C'est pourquoi, par exemple, les interpr\'etations doivent \^etre soumises \`a la m\^eme licence ou une licence compatible. De m\^eme, l'int\'egration de l'\oe uvre \`a une base de donn\'ees, une compilation ou une anthologie ne doit pas faire obstacle \`a la jouissance de l'\oe uvre telle que d\'efinie par cette licence.

\subsection{L' INTEGRATION DE L'OEUVRE}
Toute int\'egration de cette \oe uvre \`a un ensemble non soumis \`a la LAL doit assurer l'exercice des libert\'es conf\'er\'ees par cette licence.

Si l'\oe uvre n'est plus accessible ind\'ependamment de l'ensemble, alors l'int\'egration n'est possible qu'\`a condition que l'ensemble soit soumis \`a la LAL ou une licence compatible.

\subsection{CRITERES DE COMPATIBILIT\'E}
Une licence est compatible avec la LAL si et seulement si :
\begin{itemize}
\item elle accorde l'autorisation de copier, diffuser et modifier des copies de l'\oe uvre, y compris \`a des fins lucratives, et sans autres restrictions que celles qu'impose le respect des autres crit\`eres de compatibilit\'e ;
\item elle garantit la paternit\'e de l'\oe uvre et l'acc\`es aux versions ant\'erieures de l'\oe uvre quand cet acc\`es est possible ;
\item elle reconna\^it la LAL \'egalement compatible (r\'eciprocit\'e) ;
\item elle impose que les modifications faites sur l'\oe uvre soient soumises \`a la m\^eme licence ou encore \`a une licence r\'epondant aux crit\`eres de compatibilit\'e pos\'es par la LAL.
\end{itemize}

\subsection{VOS DROITS INTELLECTUELS}
La LAL n'a pas pour objet de nier vos droits d'auteur sur votre contribution ni vos droits connexes. En choisissant de contribuer \`a l'\'evolution de cette \oe uvre commune, vous acceptez seulement d'offrir aux autres les m\^emes autorisations sur votre contribution que celles qui vous ont \'et\'e accord\'ees par cette licence. Ces autorisations n'entra\^inent pas un d\'esaisissement de vos droits intellectuels.

\subsection{VOS RESPONSABILITES}
La libert\'e de jouir de l'\oe uvre tel que permis par la LAL (libert\'e de copier, diffuser, modifier) implique pour chacun la responsabilit\'e de ses propres faits.

\subsection{LA DUR\'EE DE LA LICENCE}
Cette licence prend effet d\`es votre acceptation de ses dispositions. Le fait de copier, de diffuser, ou de modifier l'\oe uvre constitue une acceptation tacite.
Cette licence a pour dur\'ee la dur\'ee des droits d'auteur attach\'es \`a l'\oe uvre. Si vous ne respectez pas les termes de cette licence, vous perdez automatiquement les droits qu'elle vous conf\`ere.
Si le r\'egime juridique auquel vous \^etes soumis ne vous permet pas de respecter les termes de cette licence, vous ne pouvez pas vous pr\'evaloir des libert\'es qu'elle conf\`ere.

\subsection{LES DIFF\'ERENTES VERSIONS DE LA LICENCE}
Cette licence pourra \^etre modifi\'ee r\'eguli\`erement, en vue de son am\'elioration, par ses auteurs (les acteurs du mouvement Copyleft Attitude) sous la forme de nouvelles versions num\'erot\'ees.
Vous avez toujours le choix entre vous contenter des dispositions contenues dans la version de la LAL sous laquelle la copie vous a \'et\'e communiqu\'ee ou alors, vous pr\'evaloir des dispositions d'une des versions ult\'erieures.

\subsection{LES SOUS-LICENCES}
Les sous-licences ne sont pas autoris\'ees par la pr\'esente. Toute personne qui souhaite b\'en\'eficier des libert\'es qu'elle conf\`ere sera li\'ee directement aux auteurs de l'\oe uvre commune.

\subsection{LE CONTEXTE JURIDIQUE}
Cette licence est r\'edig\'ee en r\'ef\'erence au droit fran\c{c}ais et \`a la Convention de Berne relative au droit d'auteur.
}
\newcommand{\INPUTTEXTELICENCE}{Le spectacle qui t'est propos\'e ce soir est diffus\'e selon les termes de la Licence Art Libre 1.3 (LAL 1.3), dont voici le
pr\'eambule.

\begin{description}
    \item[Pr\'eambule :] Avec la Licence Art Libre, l'autorisation est donn\'ee de copier, de diffuser et de transformer librement les oeuvres dans le
respect des droits de l'auteur.
Loin d'ignorer ces droits, la Licence Art Libre les reconna\^it et les prot\`ege. Elle en reformule l'exercice en permettant \`a
tout un chacun de faire un usage cr\'eatif des productions de l'esprit quels que soient leur genre et leur forme d'expression.
Si, en r\`egle g\'en\'erale, l'application du droit d'auteur conduit \`a restreindre l'acc\`es aux oeuvres de l'esprit, la Licence Art
Libre, au contraire, le favorise. L'intention est d'autoriser l'utilisation des ressources d'une oeuvre ; cr\'eer de nouvelles condi-
tions de cr\'eation pour amplifier les possibilit\'es de cr\'eation. La Licence Art Libre permet d'avoir jouissance des oeuvres tout
en reconnaissant les droits et les responsabilit\'es de chacun.

    \item[Pr\'ecision :] La Licence Art Libre 1.3 (LAL 1.3) est propagative. Cela signifie que tout enregistrement, copie, modification ou diffusion
du spectacle de ce soir sera librement enregistrable, copiable, modifiable ou diffusable selon les termes de cette licence.

\end{description}

\begin{center}
\textbf{CETTE LICENCE DE REPR\'ESENTATION DU SPECTACLE DE CE SOIR S'APPLIQUE AU
SPECTACLE ET A SES PERSONNAGES ET ARTISTES.}
\end{center}

En enregistrant le concert ce soir, et/ou en prenant des photos, tu acceptes les termes de cette licence, et t'engages donc
\`a lib\'erer ton enregistrement et/ou tes photos, \`a les rendre librement diffusable, enregistrable, copiable et modifiable.
}
\newcommand{\SETLIST}{
%---SETLIST---
\subsection{\OE uvres jou\'ees}
\begin{enumerate}
    \item Bobby Boy
    \item Mais où est mon costume ?
    \item Safety Last Prelude
    \item Prose du Pépère
    \item Telescope
    \item Odessa
    \item Mazel tof
    \item Chabicho
\end{enumerate}
\subsection{Auteurs}
\begin{itemize}
    \item Jean-Emmanuel Doucet (né le 29 Mai 1991, au Mans)
    \item Nicolas Fournier (né le 20 Juin 1987, à Paris)
    \item Aurélien Roux (né le 22 Décembre 1981, à Saint-Nazaire)
    \item Jérémy Sassier (né le 27 Mars 1980, à Le Mans)
\end{itemize}
}
\newcommand{\NOMADRSALLE}{Place de la Mairie - Hédé-Bazouges}
\newcommand{\NBREPR}{1} % CHANGE-ME
\newcommand{\DATEREP}{24 Août 2014}
\newcommand{\HEUREREP}{21h40}
\newcommand{\COMPLEMENTREP}{, SET-ME-OR-NOT}
% Date limite de fourniture de la publicite
\newcommand{\DATEPUB}{1er Août 2014}
\newcommand{\NBAFFICHES}{0} % CHANGE-ME
\newcommand{\NBFLYERS}{0} % CHANGE-ME
\newcommand{\NBPRESSBOOK}{0} % CHANGE-ME
\newcommand{\NBPHOTOS}{0} % CHANGE-ME
\newcommand{\MATOS}{salle fermée proche du lieu du spectacle pour mettre les costumes et stocker les boîtes d'instruments}
\newcommand{\NBPERSASS}{n-c} % CHANGE-ME
\newcommand{\NBPERSDEBOUT}{n-c} % CHANGE-ME
\newcommand{\NBPLACEEX}{0} % CHANGE-ME
\newcommand{\NBLITS}{0} % CHANGE-ME
\newcommand{\NUITS}{0} % CHANGE-ME
\newcommand{\NBREPAS}{2} % CHANGE-ME
\newcommand{\JOURS}{1} % CHANGE-ME
\newcommand{\PRIXTRANSP}{0} % CHANGE-ME
\newcommand{\PRIXPLACEPLEIN}{0} % CHANGE-ME
\newcommand{\PRIXPLACESREDUIT}{0} % CHANGE-ME
\newcommand{\PRIXDEMANDEURSEMPLOI}{0} % CHANGE-ME
\newcommand{\CATEGORIESREDUIT}{n-c} % CHANGE-ME
\newcommand{\PRIXPRESTA}{800} % CHANGE-ME
\newcommand{\PAIEMENTPRESTA}{par virement bancaire \`a l'ordre de l'AMMD dans les 30 jours suivant la r\'eception de la facture} % 'cheque' ou 'le texte a afficher'
\newcommand{\DATEMONTAGE}{24 Août 2014}
\newcommand{\HEUREMONTAGE}{11h00}
\newcommand{\DATEDEMONTAGE}{24 Août 2014}
\newcommand{\HEUREDEMONTAGE}{23h30}
\newcommand{\NBEX}{2} % CHANGE-ME

\newcommand{\TECHBAND}{
L'\'equipe au complet de \emph{\BANDNAME{}} pour le spectacle \emph{\NOMSPECTACLE{}} est de \NBBANDPERS{} personne\ifthenelse{\NBBANDPERS>1}{s}{}.
Se r\'ef\'erer \`a la fiche technique.
}

%%% Local Variables: 
%%% mode: latex
%%% TeX-master: "main"
%%% End: 


% Titre
\newcommand{\TitreDocument}{Contrat de cession du droit d'exploitation d'un spectacle}
\title{\TitreDocument{} \no \ContratNumero{}}
\author{\TourneurRaisonSociale}

%========================================
% LE CONTRAT
%========================================
\newcommand{\ContratNombreExemplaires}{2}
\newcommand{\ContratNumero}{\RepresentationAnnee{}\RepresentationMois{}\RepresentationJour{}-\RepresentationLieuCodePostal{}}

% Tripartite
\newcommand{\PR}{\textbf{\textsc{Le Tourneur}}}
\newcommand{\duPR}{\textbf{\textsc{du Tourneur}}}
\newcommand{\auPR}{\textbf{\textsc{au Tourneur}}}
\newcommand{\OR}{\textbf{\textsc{L'Organisateur}}}
\newcommand{\AR}{\textbf{\textsc{L'Artiste}}}

% Chemin des images
%\graphicspath{{Techbands/\GroupeNom{}}}

% R\'e\'ecriture des sections
\makeatletter
\renewcommand\section{\@startsection{section}{1}{\z@}%
	{0.6cm}%
	{0.2cm}%
	{\noindent\large\bfseries\scshape Article }}
\renewcommand\subsection{\@startsection{subsection}{2}{1em}%
	{0.24cm}%
	{-0.5cm}%
	{\bfseries\scshape}}
\makeatother
\renewcommand\thesubsection{\thesection.\Roman{subsection})}

% Ent\^ete et pied de page
\lhead{\TourneurRaisonSociale{} -- Cces\ContratNumero{} -- \GroupeNom{}}
\chead{}
\rhead{\today}
\lfoot{Paraphe \duPR\ :\newline Paraphe de \OR\ :}
\cfoot{}
\rfoot{\TitreDocument{} -- \thepage/\pageref{LastPage}}
\pagestyle{fancy}

% Document
\begin{document}
\maketitle
\thispagestyle{fancy} {\noindent\large\textbf{\textsc{Entre les
      soussign\'es}} \vspace{0.3cm} }

\noindent\emph{{\large \TourneurRaisonSociale{}}}\newline
\TourneurAdresse{},\newline \TourneurCodePostal{} \TourneurVille{} -
\TourneurPays{}\newline \emph{\TourneurTelephone{}}\newline
\TourneurNumeroSIRET{}\newline \TourneurLicencesEntrepreneur{}\newline
Repr\'esent\'e par \emph{\TourneurRepresentantNom}, en sa qualit\'e de \emph{\TourneurRepresentantQualite{}}.\\
Ci-apr\`es d\'enomm\'e \PR\ d'une part,\\

et\\

\noindent\emph{\large \OrganisateurRaisonSociale{}}\newline
\OrganisateurAdresse{},\newline \OrganisateurCodePostal{}
\OrganisateurVille{} - \OrganisateurPays{}\newline
\emph{\OrganisateurTelephone{}}\newline
\OrganisateurNumeroSIRET{}\newline
\OrganisateurLicencesEntrepreneur{}\newline
\noindent Repr\'esent\'e par \emph{\OrganisateurRepresentantNom{}}, en sa qualit\'e de \emph{\OrganisateurRepresentantQualite{}}.\\
Ci-apr\`es d\'enomm\'e \OR\ d'autre part.\\
\vspace{.5cm}
\begin{center}
    \large \textbf{\textsc{Il est expos\'e ce qui suit}}
\end{center}

\PR\ dispose des accords et d\'erogations des ayants--droits
\emph{\OrganisateurPays{}} sur tout le contenu des \oe uvres
interpr\'et\'ees dans le cadre du spectacle suivant, pour lequel il
s'est assur\'e le concours des artistes et des intervenants
n\'ecessaires \`a sa pr\'esentation au public :
\begin{center}
    \textbf{\textsc{\SpectacleNom{}}}
\end{center}
Les accords et d\'erogations sont d\'efinies suivant la licence
\emph{\SpectacleLicenceLibreNom{}}. Un exemplaire de cette licence est
disponible en annexe \vref{sec:licence} du pr\'esent contrat.

\OR\ d\'eclare conna\^itre et accepter le contenu du spectacle
pr\'e-cit\'e.  \OR\ s'est assur\'e de la disponibilit\'e de la salle
\emph{\RepresentationLieuNom{}, \RepresentationLieuAdresse{},
  \RepresentationLieuCodePostal{} \RepresentationLieuVille{} -
  \RepresentationLieuPays{}} dont \PR\ d\'eclare avoir pris
connaissance et accepter les caract\'eristiques techniques.

\vspace{.5cm}
\begin{center}
    \large \textbf{\textsc{Ceci expos\'e, il est convenu ce qui suit}}
\end{center}

\section{Objet}
\label{sec:objet}
\PR\ s'engage \`a donner, dans les conditions d\'efinies ci--apr\`es,
et dans le cadre du pr\'esent contrat de cession du droit
d'exploitation du spectacle \emph{\RepresentationNombre{}}
repr\'esentation\ifthenelse{\RepresentationNombre>1}{s}{} sur le lieu
pr\'e-cit\'e, le
\emph{\RepresentationJour{}/\RepresentationMois{}/\RepresentationAnnee{}}
\`a
\emph{\RepresentationHeure{}}\RepresentationInformationsComplementaires{}.

\section{Droits d'auteurs et droits voisins}
\label{sec:drm}
Les accords et d\'erogations d'exploitation de l'image, du logo et des
productions sonores et visuelles du spectacle, de ses personnages, et
des artistes le pr\'esentant sont d\'efinies suivant la licence
\emph{\SpectacleLicenceLibreNom{}}.

\section{Obligations \duPR}
\label{sec:oblig_pr}

\subsection{Spectacle}
\label{sec:oblig_pr:sub:spectacle}
\PR\ fournira le spectacle, d'une dur\'ee d'environ
\emph{\SpectacleDureeMinutes{}} minutes hors entracte et premi\`ere
partie (s'il y a lieu), enti\`erement mont\'e et assumera la
responsabilit\'e artistique de la repr\'esentation.

En sa qualit\'e d'employeur, il assurera les r\'emun\'erations,
charges sociales et fiscales comprises, de son personnel attach\'e au
spectacle.

\subsection{Mat\'eriels et fournitures}
\label{sec:oblig_pr:sub:matos_fournitures}
\PR\ fournira tous les \'el\'ements de d\'ecors, costumes, meubles et
accessoires, et d'une mani\`ere g\'en\'erale, tous les \'el\'ements
artistiques n\'ecessaires \`a la repr\'esentation du spectacle autres
que ceux \'eventuellement mis \`a la charge de \OR\ \`a
l'Art. \vref{sec:oblig_or:sub:gen} du pr\'esent contrat.

\subsection{Transports}
\label{sec:oblig_pr:sub:transports}
\PR\ se charge de choisir et d'organiser le transport des personnes et
du mat\'eriel attach\'es au spectacle et effectuera les \'eventuelles
formalit\'es douani\`eres dont il supportera le co\^ut. Les frais
financiers aff\'erents seront \`a la charge de \OR\ selon les termes
de l'Art. \vref{sec:heb_rest_transp} du pr\'esent contrat.

\subsection{Conditions techniques}
\label{sec:oblig_pr:sub:cond_techniques}
\PR\ fournira en annexe \vref{sec:techband} du pr\'esent contrat un
avenant pr\'ecisant les conditions techniques g\'en\'erales
pr\'evisionnelles et les conditions de cantine et de restauration de
son personnel sur le lieu de repr\'esentation. \OR\ d\'eclare en avoir
pris connaissance et en accepter l'ensemble des clauses.


\subsection{Publicit\'e}
\label{sec:oblig_pr:sub:publicite}
\PR\ fournira au plus tard le
\emph{\PrestationPubliciteDateLimiteJour{}/\PrestationPubliciteDateLimiteMois{}/\PrestationPubliciteDateLimiteAnnee{}}
les \'el\'ements n\'ecessaires \`a la publicit\'e du spectacle et
notamment
\begin{itemize}
    \item \PrestationPubliciteNombreAffiches{}
    affiche\ifthenelse{\PrestationPubliciteNombreAffiches>1}{s}{},
    \ifthenelse{\PrestationPubliciteNombreFlyers>0}{\item \PrestationPubliciteNombreFlyers{}
      flyer\ifthenelse{\PrestationPubliciteNombreFlyers>1}{s}{},}{}
    \item \PrestationPubliciteNombrePressBook{}
    dossier\ifthenelse{\PrestationPubliciteNombrePressBook>1}{s}{} de
    presse,
    \item \PrestationPubliciteNombrePhotos{}
    photo\ifthenelse{\PrestationPubliciteNombrePhotos>1}{s}{}.
\end{itemize}

\subsection{Promotion}
\label{sec:oblig_pr:sub:promo}
\PR\ s'engage \`a communiquer, dans les meilleurs d\'elais, les
accords promotionnels de ses partenaires media. \OR\ sera tenu de se
conformer aux accords conclus entre \PR\ et ses partenaires media.

\subsection{S\'ecurit\'e}
\label{sec:oblig_pr:sub:secu}
\PR\ s'engage \`a respecter et/ou \`a faire respecter la l\'egislation
et la r\'eglementation en vigueur relatives \`a la s\'ecurit\'e du
spectacle qu'il fournit.

\subsection{TVA}
\label{sec:oblig_pr:sub:tva}
Sociale ou pas, \PR\ n'est pas assujetti \`a la TVA (Art. 261-7-1e du
CGI).


\section{Obligations de \OR}
\label{sec:oblig_or}

\subsection{G\'en\'eralit\'es}
\label{sec:oblig_or:sub:gen}
\OR\ fournira le lieu de repr\'esentation en ordre de marche, y
compris le personnel n\'ecessaire au d\'echargement et au
rechargement, au montage et au d\'emontage et au service des
repr\'esentations. Il fournira le mat\'erial suivant : \RepresentationMateriel{}

Il assurera, en outre, le service g\'en\'eral du lieu : location,
accueil, billetterie, encaissement et comptabilit\'e des recettes et
service de s\'ecurit\'e \'eventuel, en se conformant \`a la
l\'egislation et \`a la r\'eglementation en vigueur.

Il s'acquittera de toutes les d\'eclarations et taxes aff\'erentes au
lieu de repr\'esentation.

En sa qualit\'e d'employeur, il assumera les r\'emun\'erations,
charges sociales et fiscales comprises, de son personnel.

Le lieu de repr\'esentation ne pourra \^etre modifi\'e par \OR\ sans
l'accord \'ecrit \duPR .

\subsection{Jauge}
\label{sec:oblig_or:sub:jauge}
\OR\ s'engage \`a ce que le nombre des spectateurs admis dans ce lieu
soit strictement inf\'erieur aux quota d\'efinis dans les
prescriptions de s\'ecurit\'e d\'etermin\'ees par la commission de
s\'ecurit\'e comp\'etente, soit
\ifthenelse{\RepresentationLieuNombrePlacesAssises>0}
{\emph{\RepresentationLieuNombrePlacesAssises{}} places assises} {}
\ifthenelse{\RepresentationLieuNombrePlacesAssises>0\AND\RepresentationLieuNombrePlacesDebout>0}
{ et } {} \ifthenelse{\RepresentationLieuNombrePlacesDebout>0}
{\emph{\RepresentationLieuNombrePlacesDebout{}} places debouts} {}.

D'une mani\`ere g\'en\'erale, il s'engage \`a respecter et/ou \`a
faire respecter la l\'egislation et la r\'eglementation en vigueur
relatives \`a la s\'ecurit\'e.

\subsection{Billetterie}
\label{sec:oblig_or:sub:billetterie}

\OR\ sera responsable de l'\'etablissement de la billetterie et en
supportera le co\^ut. Il sera \'egalement responsable de sa mise en
vente, de l'encaissement de la recette correspondante et de la mise en
place des services et personnels de contr\^ole.

Dans le cas o\`u l'image et/ou le logo du spectacle, de ses
personnages ou de ses artistes serait reproduit sur les billets, les
conditions de propagation de la \emph{\SpectacleLicenceLibreNom{}}
s'appliqueront aux visuels des billets.

% Il semblerait que cet article ne serve pas puisque l'autorisation
% est donnee par la licence d'entrepreneur du spectacle.
% \subsection{Autorisations}
% \label{sec:oblig_or:sub:autorisations}
% \OR\ sera responsable de la demande et de l'obtention des
% \'eventuelles autorisations administratives relatives \`a la
% repr\'esentation. Il communiquera \auPR\ lesdites autorisations avant
% le spectacle. Il s'assurera, par ailleurs, de la mise en place des
% services de secours m\'edical et, dans le cas d'un spectacle en
% ext\'erieur, d'am\'enagements de la circulation automobile.

% \subsection{Service de s\'ecurit\'e}
% \label{sec:oblig_or:sub:serv_secu}
% \OR\ s'engage \`a mettre en place un service de s\'ecurit\'e en
% fonction du lieu de spectacle et des perturbations susceptibles de se
% produire \`a l'occasion de la repr\'esentation.

% \OR\ devra veiller \`a ce que les membres de son service d'ordre
% r\'eservent le meilleur accueil au public et n'usent de la force qu'en
% cas de danger manifeste envers les spectateurs, le personnel du
% spectacle ou les artistes. \PR\ se r\'eserve le droit d'interrompre ou
% d'annuler une repr\'esentation s'il est t\'emoin d'une agression
% injustifi\'ee de la part d'un membre du service d'ordre.

\subsection{Publicit\'e}
\label{sec:oblig_or:sub:publicite}
En mati\`ere de publicit\'e, \OR\ s'efforcera de respecter l'esprit
g\'en\'eral de la documentation fournie par \PR\ et observera
scrupuleusement les mentions obligatoires. Dans le cas o\`u l'image
et/ou le logo du spectacle, de ses personnages ou de ses artistes
serait reproduit sur l'un des visuels de publicit\'e, les conditions
de propagation de la \emph{\SpectacleLicenceLibreNom{}} s'appliqueront
auxdits visuels.

\subsection{Promotion}
\label{sec:oblig_or:sub:promotion}
Aucune enseigne de partenaires m\'ediatiques ou commerciaux ne pourra
appara\^itre devant et dans le lieu de repr\'esentation, et en
particulier sur la sc\`ene ou sur les enceintes de diffusion.


\ifthenelse{\RepresentationLieuNombrePlacesExonerees>0}
{\subsection{Invitations}
  \label{sec:oblig_or:sub:invit}
  \OR\ s'engage \`a mettre \`a la disposition \duPR\
  \emph{\RepresentationLieuNombrePlacesExonerees{}}
  place\ifthenelse{\RepresentationLieuNombrePlacesExonerees>1}{s}{}
  exon\'er\'ee\ifthenelse{\RepresentationLieuNombrePlacesExonerees>1}{s}{}
  pour \emph{\ifthenelse{\RepresentationNombre>1}{chacune des}{la}}
  repr\'esentation\ifthenelse{\RepresentationNombre>1}{s}{}
  objet\ifthenelse{\RepresentationNombre>1}{s}{} du pr\'esent
  contrat.}{}

\section{H\'ebergement -- Restauration -- Transports}
\label{sec:heb_rest_transp}
Les frais d'h\'ebergement, de restauration et de transport seront \`a
la charge de \OR , suivant les modalit\'es suivantes :
\begin{itemize}
    \item [\emph{H\'ebergement}]
    \ifthenelse{\RepresentationLieuNombreLits>0}{\emph{\RepresentationLieuNombreLits{}
        lit\ifthenelse{\RepresentationLieuNombreLits>1}{s}{}} pour
      \emph{\RepresentationLieuNombreNuits{}
        nuit\'ee\ifthenelse{\RepresentationLieuNombreNuits>1}{s}{}}}{pas
      de prise en charge},
    \item [\emph{Restauration}]
    \ifthenelse{\RepresentationLieuNombreRepasQuotidiens>0}{\emph{\RepresentationLieuNombreRepasQuotidiens{}}
      repas
      chaud\ifthenelse{\RepresentationLieuNombreRepasQuotidiens>1}{s}{}
      complet\ifthenelse{\RepresentationLieuNombreRepasQuotidiens>1}{s}{}\footnotemark[1]
      par personne (entr\'ee, plat principal, fromage, dessert,
      boissons) sur \emph{\RepresentationLieuNombreJours{}}
      jour\ifthenelse{\RepresentationLieuNombreJours>1}{s}{}}{pas de prise en
      charge}\footnotetext[1]{selon les conditions de caterings :
      v\'eg\'etarien},
    \item [\emph{Transports}] \ifthenelse{\PrestationTransportCout>0}{location de
      véhicule + frais d'esssence et de p\'eage aller/retour pour le
      trajet \emph{\PrestationTransportCout{} \euro}}{pas de prise en charge}.
\end{itemize}

\section{Prix des places}
\label{sec:prix_places}
Les parties conviennent de fixer le prix des places comme suit
\begin{itemize}
    \item Plein tarif : \emph{\PrestationTarifPlein{} \euro}
    \item Tarif r\'eduit : \emph{\PrestationTarifReduit{} \euro}
    \item Demandeur d'emploi : \emph{\PrestationTarifDemandeurEmploi{}
      \euro}
\end{itemize}
Il est entendu que le tarif r\'eduit sera appliqu\'e aux cat\'egories
de personnes suivantes : \emph{\PrestationTarifCategoriesReduit{}}

\section{Prix de la prestation}
\label{sec:prix_presta}
\OR\ s'engage \`a verser \auPR , en contrepartie de la pr\'esente
cession, sur pr\'esentation de facture, une somme totale de
\emph{\PrestationPrixVente{} \euro} net de taxes.

\section{Modalit\'es de paiement}
\label{sec:modalite_paiement}
Le r\`eglement des sommes pr\'evues \`a l'Art. \vref{sec:prix_presta}
sera effectu\'e \ifthenelse{\equal{\PrestationMoyenPaiement}{cheque}}{le jour
  du spectacle avant la repr\'esentation par ch\`eque bancaire ou
  postal}{\PrestationMoyenPaiement{}}.

\section{Ventes annexes}
\label{sec:ventes_annexes}
\OR\ fera son affaire des \'eventuelles ventes annexes (boissons,
restauration, \dots) et en gardera le b\'en\'efice. Toutefois, il est
convenu que les boissons pourront \^etre préférablement vendues dans
des verres en plastique ou carton, et à défaut dans des bo\^ites ou
des canettes en verre ou m\'etal. Par ailleurs, \OR\ décidera du lieu
de consommation de ces ventes annexes.

\PR\ fera son affaire des ventes li\'ees au spectacle, \`a ses
personnages et/ou \`a ses artistes ou \`a leur image (albums,
T-shirts, stickers, affiches, \dots) et en gardera le b\'en\'efice
sauf accord particulier qui devra faire l'objet d'un avenant.


\section{Montage -- D\'emontage}
\label{sec:montage_demontage}
\OR\ tiendra le lieu de spectacle \`a la disposition \duPR\ \`a partir
du
\RepresentationMontageJour{}/\RepresentationMontageMois{}/\RepresentationMontageAnnee{}
\`a \RepresentationMontageHeure{} pour permettre d'effectuer le
montage, les r\'eglages et d'\'eventuels raccords. Le d\'emontage et
le rechargement seront \'effectu\'es le
\RepresentationDemontageAnnee{}/\RepresentationDemontageAnnee{}/\RepresentationDemontageAnnee{}
\`a partir de \RepresentationDemontageHeure{}.

\section{Responsabilit\'es}
\label{sec:resp}
Chaque partie garantit l'autre partie contre tout recours des
personnels, fournisseurs et prestataires dont elle a personnellement
la charge au titre des obligations respectives d\'efinies au pr\'esent
contrat.

\section{Assurances}
\label{sec:assurances}
\PR\ est tenu d'assurer contre tous les risques pouvant subvenir \`a
l'occasion des transports et entreposages ex\'ecut\'es entre deux
repr\'esentations tout objet lui appartenant ou appartenant \`a son
personnel. Il d\'eclare en outre avoir souscrit toutes les assurances
n\'ecessaires \`a ses dispositifs techniques.

\OR\ d\'eclare avoir souscrit les assurances n\'ecessaires \`a la
couverture des risques li\'es \`a l'exploitation du spectacle dans son
lieu, notamment en mati\`ere de responsabilit\'e civile. Il mettra \`a
la disposition \duPR\ des loges fermant \`a cl\'e et sera responsable
de la protection et du gardiennage de tout objet appartenant \auPR\ ou
appartenant \`a son personnel.

Dans le cas d'un spectacle en plein air, \OR\ s'engage \`a souscrire
une assurance couvrant les risques d'intemp\'eries pour les frais
incombant \`a chacun, \'etant entendu que cette assurance n\'ecessite
une couverture de sc\`ene.

\section{Enregistrement -- Diffusion}
\label{sec:enregistrement}
Conform\'ement \`a l'Art. \vref{sec:drm} du pr\'esent contrat,
l'autorisation est donn\'ee d'enregistrer le spectacle, le reproduire,
le diffuser, etc. Les conditions de propagations de la
\SpectacleLicenceLibreNom{} s'appliqueront \`a tout enregistrement,
photo, etc. du spectacle, de ses personnages et/ou de ses
artistes. \OR\ devra mettre en \oe uvre tous les moyens et supports
d'informations n\'ecessaires pour pr\'evenir le public des conditions
d'utilisation de toute captation photographique, vid\'eo et/ou sonore
des \oe uvres interpr\'et\'ees. En particulier, il affichera aux
entr\'ees du spectacle des affiches de format A3 (minimum) contenant
le texte transmis en annexe \vref{sec:texte_licence}.

Il demeure entendu, si \PR\ envisage de proc\'eder \`a la captation et
l'exploitation d'enregistrements sonores et/ou visuels de la
repr\'esentation, qu'il sera en mesure de le faire \`a son seul
arbitre et b\'en\'efice, ce dont \OR\ le garantit, en son nom et celui
des salles retenues, ainsi que d'\'eventuels sous-traitants. \PR\ fera
alors son affaire de toutes les d\'epenses aff\'erentes \`a cette
captation.

\section{Annulation de contrat}
\label{sec:annulation}
Le pr\'esent contrat se trouverait suspendu ou r\'esolu de plein droit
et sans indemnit\'e d'aucune sorte, dans tous les cas reconnus de
force majeure.

Le d\'efaut ou le retrait des droits de repr\'esentation \`a la date
d'ex\'ecution du pr\'esent contrat entra\^inerait sa r\'esolution de
plein droit pour inex\'ecution de l'une de ses clauses essentielles.

Toute annulation du fait de l'une ou l'autre des parties dans les
quinze jours pr\'ec\'edant la date de la repr\'esentation
entra\^inerait pour la partie d\'efaillante l'obligation de verser \`a
l'autre une indemnit\'e calcul\'ee en fonction des frais effectivement
engag\'es par cette derni\`ere \`a la date de rupture du contrat.

\section{Litiges}
\label{sec:litiges}
En cas de litige sur l'interpr\'etation ou l'application du pr\'esent
contrat, les parties conviennent de s'en remettre \`a d\'efaut d'un
accord amiable, \`a l'appr\'eciation des tribunaux de compétents.

\vfill Fait \`a \TourneurVille{} (France), le \today\ en
\ContratNombreExemplaires{}
exemplaire\ifthenelse{\ContratNombreExemplaires>1}{s}{}.

\lfoot{}
\begin{minipage}[h!]{0.5\linewidth}
    \begin{center}
        \PR
    \end{center}
\end{minipage}
\begin{minipage}[h!]{0.5\linewidth}
    \begin{center}
        \OR
    \end{center}
\end{minipage}
\vfill

\newpage
\lfoot{Paraphe \duPR\ :\newline Paraphe de \OR\ :}
\appendix
% R\'e\'ecriture des sections pour les appendices
\makeatletter \renewcommand\section{\@startsection{section}{1}{\z@}%
  {1cm}%
  {0.2cm}%
  {\noindent\large\bfseries\scshape Annexe }}
\renewcommand\subsection{\@startsection{subsection}{2}{0.2em}%
  {0.24cm}%
  {0.1cm}%
  {\large\bfseries\scshape}}
\renewcommand\subsubsection{\@startsection{subsubsection}{3}{1em}%
  {0.24cm}%
  {-0.5cm}%
  {\bfseries\scshape}} \makeatother
\renewcommand\thesubsection{\arabic{subsection}}

\section{\SpectacleLicenceLibreNom{}}
\label{sec:licence}
\SpectacleLicenceLibre{}

\section{Texte \`a l'intention du public}
\label{sec:texte_licence}
\SpectacleLicenceLibreTexteReferenceSpectateur{}

\section{Conditions techniques n\'ecessaires \`a l'interpr\'etation du
  spectacle}
\label{sec:techband}
\GroupeElementsTechniques{}

% \section{Conditions techniques du lieu de repr\'esentation}
% \label{sec:techsalle}
% ---INPUTTECHSALLE---

\section{Listes des pi\`eces jou\'ees et des auteurs}
\SpectacleContenuJoue{}


\end{document}

%%% Local Variables: 
%%% mode: latex
%%% TeX-master: t
%%% End: 
