\emph{Licence Art Libre 1.3 (LAL 1.3)}

\subsection{Pr\'eambule}

Avec la Licence Art Libre, l'autorisation est donn\'ee de copier, de diffuser et de transformer librement les \\oe  uvres dans le respect des droits de l'auteur.

Loin d'ignorer ces droits, la Licence Art Libre les reconna\^it et les prot\`ege. Elle en reformule l'exercice en permettant \`a tout un chacun de faire un usage cr\'eatif des productions de l'esprit quels que soient leur genre et leur forme d'expression.

Si, en r\`egle g\'en\'erale, l'application du droit d'auteur conduit \`a restreindre l'acc\`es aux \oe uvres de l'esprit, la Licence Art Libre, au contraire, le favorise. L'intention est d'autoriser l'utilisation des ressources d'une \oe uvre ; cr\'eer de nouvelles conditions de cr\'eation pour amplifier les possibilit\'es de cr\'eation. La Licence Art Libre permet d'avoir jouissance des \oe uvres tout en reconnaissant les droits et les responsabilit\'es de chacun.

Avec le d\'eveloppement du num\'erique, l'invention d'internet et des logiciels libres, les modalit\'es de cr\'eation ont \'evolu\'e : les productions de l'esprit s'offrent naturellement \`a la circulation, \`a l'\'echange et aux transformations. Elles se pr\^etent favorablement \`a la r\'ealisation d'\oe uvres communes que chacun peut augmenter pour l'avantage de tous.

C'est la raison essentielle de la Licence Art Libre : promouvoir et prot\'eger ces productions de l'esprit selon les principes du copyleft : libert\'e d'usage, de copie, de diffusion, de transformation et interdiction d'appropriation exclusive.

\subsection{D\'efinitions}

Nous d\'esignons par \og~\oe uvre~\fg, autant l'\oe uvre initiale, les \oe uvres cons\'equentes, que l'\oe uvre commune telles que d\'efinies ci-apr\`es :

\begin{description}
    \item[L'\oe uvre commune :]
Il s'agit d'une \oe uvre qui comprend l'\oe uvre initiale ainsi que toutes les contributions post\'erieures (les originaux cons\'equents et les copies). Elle est cr\'e\'ee \`a l'initiative de l'auteur initial qui par cette licence d\'efinit les conditions selon lesquelles les contributions sont faites.

    \item[L'\oe uvre initiale :]
C'est-\`a-dire l'\oe uvre cr\'e\'ee par l'initiateur de l'\oe uvre commune dont les copies vont \^etre modifi\'ees par qui le souhaite.

    \item[Les \oe uvres cons\'equentes :]
C'est-\`a-dire les contributions des auteurs qui participent \`a la formation de l'\oe uvre commune en faisant usage des droits de reproduction, de diffusion et de modification que leur conf\`ere la licence.

    \item[Originaux (sources ou ressources de l'\oe uvre) :]
Chaque exemplaire dat\'e de l'\oe uvre initiale ou cons\'equente que leurs auteurs pr\'esentent comme r\'ef\'erence pour toutes actualisations, interpr\'etations, copies ou reproductions ult\'erieures.

    \item[Copie :]
Toute reproduction d'un original au sens de cette licence.

\end{description}

\subsection{OBJET}
Cette licence a pour objet de d\'efinir les conditions selon lesquelles vous pouvez jouir librement de l'\oe uvre.

\subsection{L'\'ETENDUE DE LA JOUISSANCE}
Cette \oe uvre est soumise au droit d'auteur, et l'auteur par cette licence vous indique quelles sont vos libert\'es pour la copier, la diffuser et la modifier.

\subsubsection{LA LIBERT\'E DE COPIER (OU DE REPRODUCTION)}
Vous avez la libert\'e de copier cette \oe uvre pour vous, vos amis ou toute autre personne, quelle que soit la technique employ\'ee.

\subsubsection{LA LIBERT\'E DE DIFFUSER (INTERPR\'ETER, REPR\'ESENTER, DISTRIBUER)}
Vous pouvez diffuser librement les copies de ces \oe uvres, modifi\'ees ou non, quel que soit le support, quel que soit le lieu, \`a titre on\'ereux ou gratuit, si vous respectez toutes les conditions suivantes :
\begin{itemize}
\item joindre aux copies cette licence \`a l'identique ou indiquer pr\'ecis\'ement o\`u se trouve la licence ;
\item indiquer au destinataire le nom de chaque auteur des originaux, y compris le v\^otre si vous avez modifi\'e l'\oe uvre ;
\item indiquer au destinataire o\`u il pourrait avoir acc\`es aux originaux (initiaux et/ou cons\'equents).
\end{itemize}

Les auteurs des originaux pourront, s'ils le souhaitent, vous autoriser \`a diffuser l'original dans les m\^emes conditions que les copies.

\subsubsection{LA LIBERT\'E DE MODIFIER}
Vous avez la libert\'e de modifier les copies des originaux (initiaux et cons\'equents) dans le respect des conditions suivantes :
\begin{itemize}
\item celles pr\'evues \`a l'article 2.2 en cas de diffusion de la copie modifi\'ee ;
\item indiquer qu'il s'agit d'une \oe uvre modifi\'ee et, si possible, la nature de la modification ;
\item diffuser cette \oe uvre cons\'equente avec la m\^eme licence ou avec toute licence compatible ;
Les auteurs des originaux pourront, s'ils le souhaitent, vous autoriser \`a modifier l'original dans les m\^emes conditions que les copies.
\end{itemize}

\subsection{DROITS CONNEXES}
Les actes donnant lieu \`a des droits d'auteur ou des droits voisins ne doivent pas constituer un obstacle aux libert\'es conf\'er\'ees par cette licence.
C'est pourquoi, par exemple, les interpr\'etations doivent \^etre soumises \`a la m\^eme licence ou une licence compatible. De m\^eme, l'int\'egration de l'\oe uvre \`a une base de donn\'ees, une compilation ou une anthologie ne doit pas faire obstacle \`a la jouissance de l'\oe uvre telle que d\'efinie par cette licence.

\subsection{L' INTEGRATION DE L'OEUVRE}
Toute int\'egration de cette \oe uvre \`a un ensemble non soumis \`a la LAL doit assurer l'exercice des libert\'es conf\'er\'ees par cette licence.

Si l'\oe uvre n'est plus accessible ind\'ependamment de l'ensemble, alors l'int\'egration n'est possible qu'\`a condition que l'ensemble soit soumis \`a la LAL ou une licence compatible.

\subsection{CRITERES DE COMPATIBILIT\'E}
Une licence est compatible avec la LAL si et seulement si :
\begin{itemize}
\item elle accorde l'autorisation de copier, diffuser et modifier des copies de l'\oe uvre, y compris \`a des fins lucratives, et sans autres restrictions que celles qu'impose le respect des autres crit\`eres de compatibilit\'e ;
\item elle garantit la paternit\'e de l'\oe uvre et l'acc\`es aux versions ant\'erieures de l'\oe uvre quand cet acc\`es est possible ;
\item elle reconna\^it la LAL \'egalement compatible (r\'eciprocit\'e) ;
\item elle impose que les modifications faites sur l'\oe uvre soient soumises \`a la m\^eme licence ou encore \`a une licence r\'epondant aux crit\`eres de compatibilit\'e pos\'es par la LAL.
\end{itemize}

\subsection{VOS DROITS INTELLECTUELS}
La LAL n'a pas pour objet de nier vos droits d'auteur sur votre contribution ni vos droits connexes. En choisissant de contribuer \`a l'\'evolution de cette \oe uvre commune, vous acceptez seulement d'offrir aux autres les m\^emes autorisations sur votre contribution que celles qui vous ont \'et\'e accord\'ees par cette licence. Ces autorisations n'entra\^inent pas un d\'esaisissement de vos droits intellectuels.

\subsection{VOS RESPONSABILITES}
La libert\'e de jouir de l'\oe uvre tel que permis par la LAL (libert\'e de copier, diffuser, modifier) implique pour chacun la responsabilit\'e de ses propres faits.

\subsection{LA DUR\'EE DE LA LICENCE}
Cette licence prend effet d\`es votre acceptation de ses dispositions. Le fait de copier, de diffuser, ou de modifier l'\oe uvre constitue une acceptation tacite.
Cette licence a pour dur\'ee la dur\'ee des droits d'auteur attach\'es \`a l'\oe uvre. Si vous ne respectez pas les termes de cette licence, vous perdez automatiquement les droits qu'elle vous conf\`ere.
Si le r\'egime juridique auquel vous \^etes soumis ne vous permet pas de respecter les termes de cette licence, vous ne pouvez pas vous pr\'evaloir des libert\'es qu'elle conf\`ere.

\subsection{LES DIFF\'ERENTES VERSIONS DE LA LICENCE}
Cette licence pourra \^etre modifi\'ee r\'eguli\`erement, en vue de son am\'elioration, par ses auteurs (les acteurs du mouvement Copyleft Attitude) sous la forme de nouvelles versions num\'erot\'ees.
Vous avez toujours le choix entre vous contenter des dispositions contenues dans la version de la LAL sous laquelle la copie vous a \'et\'e communiqu\'ee ou alors, vous pr\'evaloir des dispositions d'une des versions ult\'erieures.

\subsection{LES SOUS-LICENCES}
Les sous-licences ne sont pas autoris\'ees par la pr\'esente. Toute personne qui souhaite b\'en\'eficier des libert\'es qu'elle conf\`ere sera li\'ee directement aux auteurs de l'\oe uvre commune.

\subsection{LE CONTEXTE JURIDIQUE}
Cette licence est r\'edig\'ee en r\'ef\'erence au droit fran\c{c}ais et \`a la Convention de Berne relative au droit d'auteur.
